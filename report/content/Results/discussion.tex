\section{Discussion}\label{sec:discussion}


\subsection{Indivisible vs. Mixed Algorithm}
As mentioned in \autoref{chp:introduction}, I initially expected the indivisible algorithm to not be able to maintain its 1/2MMS guarantee, even when cutting the cake into pieces. My expectation was that as the size of the cake pieces get smaller and smaller, the indivisible algorithm will approach the results of the mixed algorithm. My intuition told me that as the number of pieces approached $\infty$, the indivisible algorithm would essentially be turned into an algorithm for divisible goods, albeit an incredibly slow one as a massive instance such as that would require immense computing power and time using the current known algorithms.

The initial results with the purely indivisible cake (\autoref{fig:indivisible_vs_mixed}) seemed to confirm my intuition, in that the algorithm mostly achieves MMS values around 0.4 MMS, and that is with the cake being only the exact sum of the indivisible items, this achieved MMS value would strictly decrease for an instance if the cake would increase in size.

The results from the next experiments were therefore somewhat surprising. Immediately after cutting the cake into $n$ pieces the indivisible algorithm achieves its guarantee, regardless of the size of the cake. We also see as for the small and medium cake both algorithm achieve the exact same $\MMS$ over 50\% of the time which likely mean they find hte exact same allocation. AS the cake gets large though, the indivisible algorithm achieves higher MMS for almost all instances, which is quite surprising as one would expect an indivisible algorithm to perform worse the larger the cake is. The same can be said for the individual cake, where the cake can have any value, which seems to be a good mix of all of the above cases, as it seems it doesn't matter as much if the perceived value of the cake is shared amongst the agents.

Even though these results initially was quite suprising. It is yet to be proven for all instances that an indivisible algorithm will maintain its 1/2MMS guarantee. We can however convince ourselves that they make sense. 

Let us first assume an instance $\sInstance$, has no items, only a cake $\sTheCake$, and the number of agents $\sNumAgents\geq2$, all the agents MMS allocations, and MMS values for $\sInstance$ will simply be their proportional piece of the cake $\sTheCake$, which the indivisible algorithm with $\sNumAgents$ pieces of cake can achieve 1/2MMS easily. Now lets add a single indivisible item $\sItem$ to this instance. If $\sValuation_\sAgent(\sItem)<\sValuation_\sAgent(\sTheCake)/\sNumAgents$ then this good does not affect the MMS value of the agent at all, as this good can simply be given to any other agent. And if $\sValuation_\sAgent(\sItem)\geq\sValuation_\sAgent(\sTheCake)/\sNumAgents$ this will increase the MMS value of the agent, but in return this item is now worth more than a piece of cake and such any agent that receives this item does no longer receive any cake and the MMS value of the agent increases to $\sValuation_\sAgent(\sTheCake)/()\sNumAgents-1)$. This still isn't a problem for the 1/2MMS guarantee though as $\sValuation_\sAgent(\sTheCake)/(\sNumAgents-1)>\frac{1}{2}\sValuation_\sAgent(\sTheCake)/\sNumAgents$ as long as $\sNumAgents>2$. and if the instance only has 2 agents, then the agent that doesn't receive the item can receive more pieces of the cake.

Intuitively it also makes sense as a small and medium cake doesn't necessarily need ot be cut into any pieces as the items will likely be of equal or higher value so the cake doesn't affect the $\MMS$ value much. For large cakes we have the opposite that this means each agents gets at least one piece of the cake, and the remaining items are them split such that the 1/2 MMS is still achieved. A more through theoretical analysis and proof of these theorems are needed, but as the results weren't expected there simply wasn't enough time for such an analysis in this project, but this will be a natural part of future work, in combination with generalizing the conepts for heterogenous cake (and by extenesion multiple cakes).

In order to generalize these findings for heterogenous cake, one could possibly utilize what is  \emph{Weighted Proportional Cake Cutting} as explained in \cite{mms}. This concept generalizes proportionality to the weighted case in cake cutting using a weight profile. This would however require some more pre-processing in order to convert to and from homogenous and heterogenous cakes, which reduces one of the main benefits of simply using a indivisible algorithm directly

Another surpising result is that there seems to be some small subset of variable instances that the mixed algorithm is able to solve better than the indivisible algorithm, despite the indivisible algorithm outperforming for all cake sizes. No further analysis was performer to extract these instances for further analysis as they represented a small percentage of the instances, they are however frequent enough that they should be investigated further. 

In \autoref{fig:other_experiments} we also see how the achieved MMS values for both algorithms has a little spike at $1-MMS$, this is likely due to the randomly created instances creating instances that are easy to give each agent exactly what they expect. This would be instances where all agents have very similar valuations for the items, which is more likely to happen as the number of goods and agents decrease. 



\subsection{Approximation vs Exact}
While the analysis of using $\MMS_\sAgent\approx\PROP_\sAgent$ to approximate the MMS value suffers from some of the same limitations as the indivisible vs mixed analysis. My results indicate, as expected that for most instances this approximation has no real effect on the resulting allocation. This is especially tru for instances with medium, and large cake, as these instances often have such an abundance of cake that it allows each agent to create an MMS allocation where all bundles have the exact same value, which will be equal to $\PROP$, clearly shown in \autoref{fig:allocation_all_mms}.

The use of an exact MMS value is done as per the described algorithm\cite{mms}, with the uncertainty that my implementation may not be the most techincally efficient, the basic runtime analysis show a massive speedup in computation time, with comparable or mostly equal results to using the exact values.


\subsection{Runtime Analysis}
Since the runtime analysis is fairly limited, I would hesitate to draw any major conclusions from their result. However since the mixed algorithm needs hundreds of times longer per instance depending on the instance than the indivisible (while cutting the cake into $n$ pieces), I would argue that the indivisible algorithm outperforms the mixed algorithm in terms of runtime. However when using the approximation of each agents MMS value instead of the exact one, this runtime decreases drastically and is now faster than the indivisible algorithm. This is again not a completely fair comparison as the indivisible algorithm used in this project also handles constraints, while the mixed algorithm does not.