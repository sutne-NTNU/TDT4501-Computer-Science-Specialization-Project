\chapter{Conclusion}\label{chp:conclusion}


The implication of the results in this report is that in order to achieve a higher MMS guarantee in the Mixed Goods setting for homogenous cake, one could possible apply a indivisible algorithm with a higher MMS-guarantee, where the highest MMS approximation guarantee so far is $\frac{3}{4}+\frac{1}{12n}$ \cite{best-mms}. The benefit here being that a very simple algorithm that performs just as well as, or close to, a complex algorithm can be considered a better, or more appliable algorithm. 

In the MMS article \cite{mms} Section 4.4, improving the MMS-guarantee of the mixed algorithm is considered by using a similar logic. However their theorem implies a that a MMS-guarantee algorithm for indivisible goods can be used to achieve a approximate similar allocation for mixed instances. My results indicate however that if such an algorithm exists, then this algorithm can be used directly so solve a mixed instance with the same MMS-guarantee. However any firm conclusions should not be drawn from the experiments presented until proper and general proof is presented.

Not many conclusions can be drawn from the runtime analysis either as this analysis is too limited. The analysis also isn't fair as the indivisible algorithm\cite{Allocations} includes handling of cardinality constraints. In order for a runtime analysis to be completely fair both/all algorithms should solve the same problem.

\clearpage
\section{Future Work}\label{sec:future-work}

For a proper analysis of these results, and for the results to be significant it should be explored wether these results can be applied to heterogenous cake instances as well, in which case these results could indicate that the best approach to finding better MMS-allocation for mixed instances would be to find better algorithms for indivisible instances.

In order for this to be achieved a few things should be done:

\begin{itemize}
    \item The algorithms should be tested on a larger number of instances, and with a more varied number of agents and goods.
    \item Proof of the resulting indications must be provided.
    \item The algorithms should be tested and applied for instances with heterogenous cake.
    \item The experiments run in this report should be repeated with indivisible algorithms with higher MMS-guarantees to confirm wether these algorithms are as stable.
    \item The code used for the experimentation and algorithms must be properly peer-reviewed for quality assurance.
\end{itemize}




