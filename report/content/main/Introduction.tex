\chapter{Introduction}
\label{chp:introduction}

This is the introduction \cite{latex2e}.
Also check out \autoref{lst:c}.

\begin{listing}
    \inputminted{c}{assets/listings/hello.c}
    \caption{Hello World in C}
    \label{lst:c}
\end{listing}


\section{Code Section}\label{sec:code-section}

\begin{listing}
    \inputminted{julia}{assets/listings/hello.jl}
    \caption{Hello World in Julia}
    \label{lst:julia}
\end{listing}

\section{Preliminaries}
\label{sec:preliminaries}

Preliminaries which also show \autoref{fig:my test graph}.

\begin{figure}
    \centering
    \begin{tikzpicture}
        \node[node] (a) {a};
        \node[node] (b) [right = of a] {b};
        \node[node] (c) [right = of b] {c};
        \node[node] (d) [below = of a] {d};
        \node[node] (e) [right = of d] {e};
        \node[node] (f) [right = of e] {f};
        \node[node] (g) [below = of d] {g};
        \node[node] (h) [right = of g] {h};
        \node[node] (i) [right = of h] {i};
        \draw[edge] (a) -- (b);
        \draw[edge] (a) -- (d);
        \draw[edge] (b) -- (c);
        \draw[edge] (c) -- (f);
        \draw[edge] (d) -- (e);
        \draw[edge] (d) -- (h);
        \draw[edge] (e) -- (b);
        \draw[edge] (e) -- (g);
        \draw[edge] (f) -- (i);
        \draw[edge] (g) -- (h);
        \draw[edge] (h) -- (f);
        \draw[edge] (e) -- (i);
    \end{tikzpicture}
    \caption{Just a graph}
    \label{fig:my test graph}
\end{figure}


And also check out the figure in \autoref{fig:my test appendix figure}.