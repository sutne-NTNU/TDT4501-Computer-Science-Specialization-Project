\section{Preliminaries}\label{sec:preliminaries}

Collection of terminology, notation and definitions that are presumed to be known for the rest of the report.


\subsection*{Mixed Goods}
When discussion mixed goods we are talking about a mix of both indivisible and divisible goods (often referred to as $cake$). A mixed goods instance can contain any combination of number of divisible and indivisible goods. A good means that all items will have a positive utility or value for all agents.



\subsection*{Problem Instance}
An instance $\langle N, M, V, F \rangle$ of the fair division problem is defined by a:
\begin{itemize}
    \item set $N$ of $n \in N$ agents
    \item set $M$ of $m \in M$ indivisible items
    \item set $V$ of $v \in V$ valuation functions where $v_i$ is valuation function for agent $i$
    \item density function $F$ that divides divisible items
\end{itemize}
a set $M$ of $m \in N$ indivisible items, and a valuation profile $V = {v_1, v_2, . . . , v_n}$ that specifies the preferences of every agent $i \in N$ over each subset of the items in $M$ via a valuation function $v_i : 2M \rightarrow R$.



\subsection*{Additive valuations}
A well-studied subclass of monotone valuations is that of additive valuations, wherein an agent's value of any subset of items is equal to the sum of the values of individual items in the set, i.e., for any agent $i \in N$ and any set of items $S \subseteq M$, $v_i(S) := j \in S vi({j})$, where we assume that $v_i(\emptyset) = 0$. For simplicity, we will write $v_i(j)$ or $v_{i,j}$ to denote $v_i({j})$.



\subsection*{Bundle}
The collection of items that an agent receives is called a bundle. A bundle is a subset of the items in $M$.



\subsection*{Allocation}
An allocation $A := (A1,...,An)$ is an $n$-partition of a subset of the set of items $M$, where $Ai \subseteq M$ is the \textit{bundle} allocated to the agent $i$ (note that $A_i$ can be empty $\emptyset$). An allocation is said to be complete if it assigns all items in $M$, and is called partial otherwise.



\subsection*{Top-trading envy graph}
The top-trading envy graph $T_A$ of an allocation $A$ is a subgraph of its envy graph $G_A$ with a directed edge from agent $i$ to agent $k$ if $v_i(A_k) = max_{j\in N} v_i(A_j)$ and $v_i(A_k) > v_i(A_i)$, i.e., if agent $i$ envies agent $k$ and $A_k$ is the most preferred bundle for agent $i$.



\subsection*{Envy Graph}
The envy graph $GA$ of an allocation $A$ is a directed graph on the vertex set $N$ with a directed edge from agent $i$ to agent $k$ if $vi(Ak) > vi(Ai)$, i.e., if agent $i$ prefers the bundle $Ak$over the bundle $Ai$.



\subsection*{Envy-freeness and its relaxations}
\textbf{EF}\\
An allocation $A$ is said to be envy-free (EF) if for every pair of agents $i,k \in N$, we have $v_i(A_i) \geq v_i(A_k)$,

\textbf{EF1}\\
An allocation $A$ is said to be envy-free up to one item (EF1) if for every pair of agents $i, k \in N$ such that $A_i \cap A_k \backslash= \emptyset$, there exists an item $j \in A_i \cup A_k$ such that $v_i(A_i \backslash {j}) \geq v_i(A_k \backslash {j})$.

\textbf{Envy-freeness for mixed goods (EFM)}\\
An allocation $A$ is said to satisfy envy-freeness for mixed goods (EFM) if for any agents $i,j \in N$,
\begin{itemize}
    \item if agent $j$'s bundle consists of only indivisible goods, there exists $g \in A_j$ such that $u_i(A_i) \geq u_i(A_j \backslash {g})$;
    \item otherwise, $u_i(A_i) \geq u_i(A_j)$.
\end{itemize}
It is easy to see that when the goods are all divisible, EFM reduces to EF; when goods are all indivisible, EFM reduces to EF1. Therefore EFM is a natural generalization of both EF and EF1 to the mixed goods setting.
